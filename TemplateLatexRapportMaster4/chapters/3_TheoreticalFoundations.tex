\chapter{Issues with the Line of Action}\label{chap:issues}

\section{Maintaining Contact}
\subsection{Ground Contact}
Contact is already an issue with one character, made worse when more than one character is involved. Contacts for one character are constraints between the character's body parts and itself or other objects in the scene, like the floor, for instance. Now when other characters are introduced to the scene, there are even more options for potential contacts between body parts of one character and body parts of another. In our case, we only have two characters, but this research could be further generalized to a larger number of characters.


\subsection{Contact Between Characters}


\section{Twisting}


%\section{Parallelism and Symmetry}

\section{Collisions}
\subsection{Self-Collisions}
Self collisions are when a part of the character's body collides with another part of its own body. 

\subsection{Collisions Between Characters}
