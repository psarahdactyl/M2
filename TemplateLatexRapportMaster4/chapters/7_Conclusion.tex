\chapter{Conclusion}\label{chap:conclusion}
The goal of the thesis was to expand upon IMAGINE's existing tools by designing and constructing the capability to animate interactions between characters with the line of action. While the capability to animate a single character by drawing a curve is present, there is no established sketch-based way to animate multiple characters either interacting or moving separately. The sketch-based animation of dancing couples captures the challenge of animating two characters both separately and together.

First I observed and analyzed the current methods of creating character animations with and without interactions, from sketch-based animation to generating animation through motion graphs made from existing data. I looked at the way artists, cartoonists, and choreographers draw lines to indicate motion, specifically that of dancers.

I have identified what I think those tools lack for the task of posing dancing characters. Based on my findings, I created an algorithm for being able to describe and reproduce the coordinated movements of two dancers during a dance step. I evaluated the current line of action tool with metrics that could later be compared to the combined kinematic tree approach.

Dance is simply an introductory example for the design and evaluation of the proposed techniques, but the methods have the possibility to be applicable more generally.