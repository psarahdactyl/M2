\documentclass[12pt, a4paper, parskip=full, dvipsnames]{memoir} % for a short document
\usepackage[french,english]{babel}

\usepackage [vscale=0.76,includehead]{geometry}                % See geometry.pdf to learn the layout options. There are lots.
% \geometry{a4paper}                   % ... or a4paper or a5paper or ... 
% \geometry{landscape}                % Activate for for rotated page geometry
% \OnehalfSpacing
% \setSingleSpace{1.05}
% \usepackage[parfill]{parskip}    % Activate to begin paragraphs with an empty line rather than an indent


%===================================== packages
\usepackage{lipsum}
\usepackage{graphicx}
\usepackage{amsmath}
\usepackage{fullpage}
\usepackage{mathptmx} % font = times
\usepackage{helvet} % font sf = helvetica
\usepackage[latin1]{inputenc}
\usepackage{relsize}
\usepackage[T1]{fontenc}
\usepackage{tikz}
\usepackage{booktabs}
\usepackage{textcomp}%textquotesingle
\usepackage{multirow}
\usepackage{pgfplots}
\usepackage{url}
\usepackage{footnote}


%\usepackage{datetime2}
\usepackage[en-US,showdow=true]{datetime2}
\usepackage{amsmath}
\usepackage{graphicx}
\usepackage{wrapfig}
\usepackage{caption}
\usepackage{subcaption}
\usepackage{natbib}
\usepackage{hyperref}
\usepackage{datatool}
\usepackage{array}
\usepackage{multibib}
\newcites{online}{Internet Sources}
\usepackage{cleveref}
\usepackage{makeidx} 
\usepackage[]{algorithm2e}
\usepackage[]{xcolor}
\usepackage{fancyvrb}
\usepackage{float}
\usepackage{adjustbox}
%============================================

% redefine \VerbatimInput
\RecustomVerbatimCommand{\VerbatimInput}{VerbatimInput}%
{fontsize=\footnotesize,
 %
 framesep=2em, % separation between frame and text
 labelposition=topline,
 %
 commandchars=\|\(\), % escape character and argument delimiters for
                      % commands within the verbatim
 commentchar=*        % comment character
}

%============================================
\usetikzlibrary{arrows,shapes,positioning,shadows,trees}
\makesavenoteenv{tabular}
\makesavenoteenv{table}
%==============================================
\def\checkmark{\tikz\fill[scale=0.4](0,.35) -- (.25,0) -- (1,.7) -- (.25,.15) -- cycle;}
%Style des têtes de section, headings, chapitre
\headstyles{komalike}
\nouppercaseheads
\chapterstyle{dash}
\makeevenhead{headings}{\sffamily\thepage}{}{\sffamily\leftmark} 
\makeoddhead{headings}{\sffamily\rightmark}{}{\sffamily\thepage}
\makeoddfoot{plain}{}{}{} % Pages chapitre. 
\makeheadrule{headings}{\textwidth}{\normalrulethickness}
%\renewcommand{\leftmark}{\thechapter ---}
\renewcommand{\chaptername}{\relax}
\renewcommand{\chaptitlefont}{ \sffamily\bfseries \LARGE}
\renewcommand{\chapnumfont}{ \sffamily\bfseries \LARGE}
\setsecnumdepth{subsection}


% Title page formatting -- do not change!
\pretitle{\HUGE\sffamily \bfseries\begin{center}} 
\posttitle{\end{center}}
\preauthor{\LARGE  \sffamily \bfseries\begin{center}}
\postauthor{\par\end{center}}
\newcommand{\jury}[1]{% 
\gdef\juryB{#1}} 
\newcommand{\juryB}{} 
\newcommand{\session}[1]{% 
\gdef\sessionB{#1}} 
\newcommand{\sessionB}{} 
\newcommand{\option}[1]{% 
\gdef\optionB{#1}} 
\newcommand{\optionB} {}

\renewcommand{\maketitlehookd}{% 
\vfill{}  \large\par\noindent  
\begin{center}\juryB \bigskip\sessionB\end{center}
\vspace{-1.5cm}}
\renewcommand{\maketitlehooka}{% 
\vspace{-1.5cm}\noindent\includegraphics[height=12ex]{pics/logo-uga.png}\hfill\raisebox{2ex}{\includegraphics[height=14ex]{pics/logoINP.png}}\\
\bigskip
\begin{center} \large
Master of Science in Informatics at Grenoble \\
Master Informatique \\ 
Specialization \optionB  \end{center}\vfill}
% =======================End of title page formatting

\option{Graphics, Vision, and Robotics} 
\title{Sketch-Based Posing and Interaction of Multiple Characters: Animating Dancing Couples} %\\\vspace{-1ex}\rule{10ex}{0.5pt} \\sub-title} 
\author{Sarah Anne Kushner}
\DTMlangsetup[en-US]{monthyearsep={\space}}
\newcommand{\mydate}{\DTMdisplaydate{2017}{06}{21}{2}}
\date{\mydate} % Delete this line to display the current date

\jury{
Research project performed at Inria Grenoble -- Rh\^one-Alpes\\\medskip
\rule{\textwidth}{1pt}

Under the supervision of:\\

%\noindent\makebox[0.8\textwidth]{Prof. Marie-Paule Cani\hfill Dr. R\'emi Ronfard}
%\noindent\makebox[0.8\textwidth]{\'Ecole Polytechnique\hfill Inria}

\begin{tabular}{c c c}
  Prof. Marie-Paule Cani & \hfill & Dr. R\'emi Ronfard \\
  \'Ecole Polytechnique & \hfill & Inria Grenoble \\
  Paris-Saclay & \hfill & Rh\^one-Alpes\\

\end{tabular}

\rule{\textwidth}{1pt}

Defended before a jury composed of:\\
Prof. James Crowley\\
Prof. Jo\"elle Thollot\\
Prof. Laurence Nigay\\
Assistant Prof. Dominique Vaufreydaz\\
}
\session{June \hfill 2017}
\setcounter{tocdepth}{4}
\setcounter{secnumdepth}{4}
\maxsecnumdepth{subsubsection}

%==============================================
\makeindex
%==============================================

\setlength{\parindent}{0pt}
\nonzeroparskip 
 
%%% BEGIN DOCUMENT
\begin{document}
\selectlanguage{English} % french si rapport en français
\frontmatter
\begin{titlingpage}
\maketitle
\end{titlingpage}

%\small
%\setlength{\parskip}{-1pt plus 1pt}

\renewcommand{\abstracttextfont}{\normalfont}
\abstractintoc
\begin{abstract} 
3D human characters are difficult to pose and animate on their own, let alone while interacting with each other. Even experienced animators require a lot of time setting poses over time to create precise animations. The line of action, an artistic tool used mainly by cartoonists to indicate expressive poses, translates well as a concept to 3D and due to previous research is, available in a convenient and intuitive sketch-based interface. With the line of action, a user can sketch a line in the shape that they want a selected body part of a  character to take. 

This technique has been proven to work well with a single character. However, extending the line of action to work for multiple characters is not a simple task. Many issues come with animating multiple characters who interact with each other. Avoiding collisions, maintaining contacts, and twisting the characters become rather difficult.

We aim to extend the line of action to handle cases involving multiple characters, with a dancing couple as a case study. Dance is a good example of interactions; there are many cases of characters touching each other and moving separately. The characters come together and move apart over the course of a dance sequence. To allow the line of action to span over multiple characters in an intuitive way, we propose an algorithm to merge the skeletons of the characters to be able to pose the characters as one.

\end{abstract}
\abstractintoc
\newpage
\renewcommand\abstractname{R\'esum\'e}
\begin{abstract} \selectlanguage{French}
Les personnages humains en 3D sont difficiles \`{a} positionner et \`{a} animer par eux-m\^{e}mes, sans parler de l'interaction des les uns avec les autres. M\^{e}me les animateurs exp\'{e}riment\'{e}s ont besoin de beaucoup de temps pour \'{e}tablir des poses au fil du temps afin de cr\'{e}er des animations pr\'{e}cises. La ligne d'action est un outil artistique utilis\'{e} principalement par les caricaturistes pour indiquer des poses expressives. Il se traduit bien par un concept \`{a} 3D et, gr\^{a}ce \`{a} des recherches ant\'{e}rieures, est disponible dans une interface bas\'{e}e sur l'esquisse pratique et intuitive. Avec la ligne d'action, un utilisateur peut esquisser une ligne d\'{e}crivant la forme souhait\'{e} que la partie partie du corps s\'{e}lectionn\'{e} d'un personnage prenne.

Cette technique s'est av\'{e}r\'{e}e bien fonctionner avec un seul personnage. Cependant, \'{e}tendre la ligne d'action pour travailler avec plusieurs personnages n'est pas une t\^{a}che simple. Beaucoup de probl\`{e}mes apparaissent dans l'animation de plusieurs personnages interagissant les uns avec les autres. Eviter les collisions, maintenir les contacts et tordre les personnages devient plut\^{o}t difficile.

Nous visons \`{a} \'{e}tendre la ligne d'action pour g\'{e}rer les cas impliquant plusieurs personnages, avec un couple dansant comme \'{e}tude de cas. La danse est un bon exemple d'interactions; Il y a plusieurs cas o\`{u} les personnages se touchent et se d\'{e}placent s\'{e}par\'{e}ment. Les personnages se rassemblent et s'\'{e}cartent au cours d'une s\'{e}quence de danse. Pour permettre \`{a} la ligne d'action de s'\'{e}tendre sur plusieurs personnages de mani\`{e}re intuitive, nous proposons un algorithme pour fusionner les squelettes des personnages afin de pouvoir positionner les personnages comme s'ils y en avait q'un seul. 
\end{abstract}
\selectlanguage{English}

\newpage

\renewcommand\abstractname{Acknowledgement}
\begin{abstract}
\newcommand{\ignore}[1]{}

I would like to officially say thank you to those without whom this thesis would not have been possible. It brings me joy to acknowledge the following people. \\

Thank you to\ignore{Associate Professor} David Breen,\ignore{Assistant Research Professor} Marcello Balduccini and\ignore{Associate Teaching Professor} Daryl Falco from Drexel University for being not only excellent and inspiring mentors and teachers, but also for helping and supporting me on my journey to study in France. \\

James Crowley deserves credit for making my move to France viable, and putting up with my numerous emails. I would also like to thank him for teaching me in the first semester of M2 along with all the other MoSIG M2 professors. Thank you to the Fondation partenariale Grenoble INP for funding my first semester.\\

I would like to express my sincere gratitude to both my supervisors Marie-Paule Cani and R\'emi Ronfard for their invaluable assistance, guidance, and patience through the process of a Master thesis.\\

To the IMAGINE engineers Maguelonne Beaud de Brive and Julien Daval: thank you for answering my development questions and teaching me the mysterious ways of Rumba and Blender. \\

I have so much appreciation for Tom. Thank you for letting me think out loud to you on the phone; comforting me when I freak out to you on the phone; sending me pictures of our baby bunny, Cinnamon Louise Bun; and providing me with constant encouragement all the way from the United States. \\

It wouldn't be fair not to mention and thank my dear dancer friends Ruth and Vivian for answering my sometimes obvious and weirdly specific questions about your experiences in dance and choreography.\\

Shout out to all the C1 bus drivers for getting me to and from Inria every day. Merci, au revoir. \\

Lastly, thank you to the jury for reading this report and listening to my presentation.


\end{abstract}



\cleardoublepage

\tableofcontents* % the asterisk means that the table of contents itself isn't put into the ToC
\normalsize

\mainmatter
\SingleSpace
%==============================CHAPTERS==================
\chapter{Introduction}\label{chap:intro}
\newcommand{\ignore}[1]{}
\ignore{Write this chapter LAST. Should be 5 to 10 pages. This chapter provides a quick summary of the
essential contents of the research project, principal results and contents of the report. The target
audience is members of the jury who do NOT have time to completely read all 21 reports, as well
academic members of other juries who wish to compare this work to other works.}

\section{Background}
\ignore{This is a generic title. Replace it with an actual title that describes the context of the work.
Short .5 page summary of the technological context of the work and why it is interesting or important}

3D animation can be a painstakingly tedious activity. To create a desired animation, animators go through the long process of keyframing. Keyframes are set positions that define the start and end points of a movement, sequences of poses which are transformed in time. Typically, animators assign poses to certain frames over time, so that in-between motions can be generated by a computer. To get an accurate animation, artists usually must assign many keyframes, then spend time adjusting and editing them to be more precise. The fact that industry professionals take so much time and effort to do this shows that for an amateur or untrained artist, creating \textit{good} 3D animation is close to impossible. 

In \autoref{fig:keyframes}, the character has keyframes attached to it on the timeline, shown in the yellow box, which represent its various positions and rotations in time.

%\begin{wrapfigure}{R}{0.5\textwidth}
\begin{figure}[h!]
\centering
\includegraphics[scale=0.7]{img/newrunningposes}
\caption{Example of keyframing in Autodesk Maya. Red lines on the timeline indicate keyframes for the character.}
\label{fig:keyframes}
\end{figure}
%\end{wrapfigure}

\section{Problem Statement}
\ignore{This is a generic title. Replace it with an actual title that describes the context of the work.
Approx .5 to 1 page description of the research problems that was addressed and what was required to address it.}

Among the most complicated characters to animate in 3D animation are humanoid characters. To ease this task, animators create a skeleton for their character called a \textbf{rig}, that consists of joints connected by rigid links (bones) to give a structure to the character. Humanoid rigs can range in complexity from somewhat simple to extremely complicated depending on the amount of detail desired by the user. The structure is a hierarchy of joints that can also be seen as a tree with a root, which in the humanoid case, is usually the pelvis. The leaf nodes of this tree, which are loacted at the maximal parts of the body, are called \textbf{end effectors}. Leaf nodes come at the end of a \textbf{kinematic chain}, which can be followed back up to the root.


\begin{figure}[h!]
	\centering
        \begin{subfigure}[b!]{0.55\textwidth}
        	\centering
                \includegraphics[width=\linewidth]{img/skeleton}
                \caption{Example of a humanoid skeleton.}
                \label{fig:skeleton}
        \end{subfigure}
        \quad
        \begin{subfigure}[b!]{0.4\textwidth}
        	\centering
                \includegraphics[width=\linewidth]{img/skeleton_hierarchy}
                \caption{The hierarchy of joints corresponding to the skeleton.}
                \label{fig:hierarchy}
        \end{subfigure}%
	\caption{Humanoid skeleton shown in Blender.}
	\label{fig:rig}
\end{figure}

\subsection{Kinematics}
Forward and inverse kinematics are two general animation methods used mainly in situations in which articulated characters need to move according to some contraints. In order to animate this structure successfully, controls are added that allow for forward and inverse kinematics. These controls help the animator move the character into poses that will then act as keyframes.

\textbf{Forward kinematics} (FK) is a method of calculating the position and orientation of the end effector (i.e. a hand or foot) given the positions and angles of the joints higher up in the chain all the way to the root. 

\textbf{Inverse kinematics} (IK) is the method opposite of forward kinematics. That is, the goal is to calculate the angles and positions of joints in the chain, given the angle and position of the end effector. This goal is much harder to reach, seeing that more information needs to be calculated than is given. This problem is underconstrained. There can be more than one correct configuration that satisfies the constraints or there can even be no viable configurations. Many IK algorithms exist to calculate joint angles and positions, which will be explained further in \autoref{chap:theory}.

Inverse kinematics are clearly more desirable for an animator since it is easier and faster to pose a character and have the joint angles automatically computed than it is to manipulate the character's joints directly.

\subsection{Multiple Characters}
Although animation using kinematic controls is the standard, sketch-based interfaces have started to become a plausible option for both animators and those with less experience. Even manipulating controls can be time-consuming. Many papers have been authored dealing with the sketch-based posing of a single humanoid character, discussed later in \autoref{chap:sota}. However, the animation of multiple characters comes with its own unique set of challenges as well. The problem is discovered when humanoid characters interact, namely when they are in close proximity to each other or when they touch each other.

Collisions and contact are both already issues with one character, made worse when more than one character is involved. Self collisions are when a part of the character's body collides with another part of its own body. Contacts for one character are constraints between the character's body parts and itself or other objects in the scene, like the floor, for instance. Now when other characters are introduced to the scene, there are even more options for potential collisions and contacts between body parts of one character and body parts of another. In our case, we only have two characters, but this research could be further generalized to a larger number of characters.

\subsection{Posing}
The line of action is the concept of imagining a line that extends through the character's main action. It is commonly used by cartoonists in gesture drawings and the early stages of storyboarding to accentuate the motion and shape of the character. These lines are often dramatic in shape but smooth and simple in quality, usually containing only one or two extrema. The line of action goes through the majority of a character's body or through a part of the body and has a clear direction. See \autoref{fig:lines}.

\begin{figure}[h!]
	\centering
        \begin{subfigure}[b!]{0.45\textwidth}
        	\centering
                \includegraphics[width=\linewidth]{img/cartoon}
                \label{fig:gesture}
        \end{subfigure}
        \quad
        \begin{subfigure}[b!]{0.45\textwidth}
        	\centering
                \includegraphics[width=\linewidth]{img/kick}
                \label{fig:kick}
        \end{subfigure}%
        \caption{Line of action examples.}
	\label{fig:lines}
\end{figure}

\section{Scientific Approach to Posing and Animating Multiple Characters}
\ignore{This is a generic title. Replace it with an actual title that describes the context of the work.
Approx 1 to 2 page description of the scientific approach or approaches to a solution and how it was investigated and evaluated. Present a summary of the principal results obtained.}

The use case for this research in multi-character animation is a dancing couple. There are many combinations of poses a human can be in, let alone two humans \textit{and} the two humans interacting. Specifically, a clip from the film ``The Band Wagon'' (\citep{thebandwagon1953}) was used to observe the common motions and poses in a dancing couple.

Almost inverting the problem at hand, we start from a pose to find the appropriate lines of action by tracing over the shapes of the dancers' bodies using the Grease Pencil Tool in Blender. It was carried out with the purpose of clarifying the properties and limitations of the typical line of action. After seeing at which point the actions started to repeat, a shorter sub-clip was chosen. Annotation was performed in five different ways:
\begin{enumerate}
	\item tracking and marking each characters' contact with the ground
	\item tracking and marking each characters' contact with each other
	\item tracing lines of action for each character separately every few frames, using the previous line of action notation -- the baseline method.
	\item tracing lines of action for the characters and treating them as one using as few strokes as possible every few frames.
	\item once the main common combinations were established from the above two methods, tracing lines of action only for the selected extreme combination poses.
\end{enumerate}

\begin{figure}[h!]
	\centering
        \begin{subfigure}[b!]{0.45\textwidth}
        	\centering
                \includegraphics[width=\linewidth]{img/keyframe_case_7_baseline}
                \caption{The typical lines of action trace out the whole skeleton, and in the worst case is 3 lines per character.}
                \label{fig:baseline}
        \end{subfigure}
        \quad
        \begin{subfigure}[b!]{0.45\textwidth}
        	\centering
                \includegraphics[width=\linewidth]{img/keyframe_case_7_new}
                \caption{Treating the characters as one bigger character allows the user to draw fewer lines to indicate the same pose.}
                \label{fig:new_notation}
        \end{subfigure}%
        \caption{Comparing the old line of action and the new.}
	\label{fig:poses}
\end{figure}

Through this process, certain patterns and common poses stood out, enabling the development of an improved notation for illustrating the poses of characters during their interactions with each other. Patterns included symmetry, parallelism, and repetition. Of course, using these patterns and assumptions to our advantage is what drove the development of a solution. It seemed that during many instances in the clip, it would be more efficient to draw lines over the two dancers as if they were one character. This morphed ``combined'' character should be easier to pose and animate.

\begin{figure}[h!]
	\centering
        \begin{subfigure}[b!]{0.45\textwidth}
        	\centering
                \includegraphics[width=\linewidth]{img/parallelism}
                \caption{Parallelism.}
                \label{fig:parallelism}
        \end{subfigure}
        \quad
        \begin{subfigure}[b!]{0.45\textwidth}
        	\centering
                \includegraphics[width=\linewidth]{img/symmetry}
                \caption{Symmetry.}
                \label{fig:symmetry}
        \end{subfigure}%
        \caption{Patterns in the poses.}
	\label{fig:patterns}
\end{figure}

\begin{figure}[h!]
	\centering
        \begin{subfigure}[b!]{0.31\textwidth}
        	\centering
                \includegraphics[width=\linewidth]{img/keyframe_case_4_(3)}
                \label{fig:pose1}
        \end{subfigure}
        \quad
        \begin{subfigure}[b!]{0.31\textwidth}
        	\centering
                \includegraphics[width=\linewidth]{img/keyframe_case_10_(5)}
                \label{fig:pose2}
        \end{subfigure}%
        \quad
        \begin{subfigure}[b!]{0.31\textwidth}
        	\centering
                \includegraphics[width=\linewidth]{img/keyframe_case_7_(4)}
                \label{fig:pose3}
        \end{subfigure}%
	\caption{A selection of tracings over the dance clip from ``The Band Wagon.''}
	\label{fig:tracing}
\end{figure}

Using the previous work (\citep{guay2013line}, \citep{guay2015adding}, and \citep{guay2015space}), already in the process of being developed by the IMAGINE team, I extend the functionality of the line of action to be applied to multiple characters. Their software acted as the baseline with which to compare the new features. Baseline poses and animations were made by taking important keyframes from the film clip to recreate in their software. The amount of time spent running the program, the number of clicks, and number of lines drawn were recorded.

Since characters' skeletons are essentially trees, a novel approach to solving this problem was combining these kinematic trees into one in a new data structure with special attributes for posing and animating. An interesting way to evaluate this notion of animating a combined character was to, again, reproduce the same poses and animations as in the baseline, but this time using the new structure instead.

Results to come I hope

\section{Contents of this report}
\ignore{Approx .5 page per chapter. Summarize the contents of the subsections of each chapter. Give the
topics addressed and summarize what is written in each chapter.}

In the following chapter (\autoref{chap:sota}), I cover the state-of-the-art for this particular problem. I discuss a brief history of dance notation -- how choreographers and dancers use sketching on paper to brainstorm and communicate their ideas of motion, formation, and pose of dance. Then I will talk about the existing sketch-based systems used for posing articulated characters, covering the benefits and limitations of each. A sizeable portion of my work has had to do with kinematic trees and graph data structures, so I will go into a few important graph theory algorithms. Generating animation from existing data is also a widely relevant topic to my work. So, I also describe previous research regarding that.

In \autoref{chap:theory}, the fundamentals of character animation are more closely examined. (more to come when I actually write it...)

\autoref{chap:implementation} covers exactly how a solution was reached and what it entails. (more to come when I actually write it...)

\autoref{chap:results} is where I go over the methods of validating our solution. (more to come when I actually write it...)

Discussion of the lessons learned during this project and the concluding thoughts on the process are explored in \autoref{chap:discussion} and \autoref{chap:conclusion}, respectively. (more to come when I actually write it...)


\chapter{State-of-the-Art}\label{chap:sota}
\ignore{

\subsection{analyzing/classification}
Analysis of impression of robot bodily expression\\
Convolutional Pose Machines

\subsection{math/algorithms}
The Conjugate Residual Method for Constrained Minimization Problems -- 2015\\
Constrained Closed Loop Inverse Kinematics -- 2010
}



\section{Dance}
Sutton Dance Writing\\
Labanotation\\
Benesh Movement Notation

\section{Sketch-Based Systems}
The IMAGINE group at Inria has made it their mission to tackle this problem. They have made significant progress on a project where they aim to offer more intuitive tools to author 3D digital content. The IMAGINE team has invented (1) a type of notation made especially for posing and animating 3D characters (2) a technique for posing called the line of action, in which a user can draw a line in the shape they want a kinematic chain to take and (3) a technique for animation called space-time sketching, in which a user can draw a line in the path they want a model to take and it will be animated accordingly. As the character follows the path, its model bends and changes shape in a physically realistic way. Their system currently supports creating different movements with the path such as bouncing, rolling, and twisting.

The line of action technique works extremely well for a single humanoid character, and even multiple humanoid characters separate from each other. 

\begin{figure}[!h]
\includegraphics[scale=0.4]{img/baseline}
\caption{One character's keyframes using the line of action technique from \citep{guay2013line}.}
\end{figure}

The Line of Action: an Intuitive Interface for Expressive Character Posing -- 2013\\
Adding dynamics to sketch-based character animations -- 2015\\
Space-time sketching of character animation -- 2015\\
\\
Artist-oriented 3D character posing from 2D strokes -- 2016\\
people in Switzerland also did the posing \\
Sketch to pose in Pixar's presto animation system -- 2015


\section{Graph Theory}
Finding All the Elementary Cycles in a Directed Graph\\
A New Search Algorithm for Finding the Simple Cycles of a Finite Directed Graph\\
An Algorithm for Combining Graphs Based on Shared Knowledge

\section{Generating Animation}
%\subsection{cleaning/editing animation}
FootSee: an Interactive Animation System\\
Footskate Cleanup for Motion Capture Editing\\
SketchiMo: Sketch-based Motion Editing for Articulated Characters -- 2016\\
sketching for editing trajectories and poses

%\subsection{retargeting motion}
Retargetting Motion to New Characters\\
Using an Intermediate Skeleton and Inverse Kinematics for Motion Retargeting

%\subsection{generating animation}
Motion Graphs\\
Style-Based Inverse Kinematics -- 2004\\
generative models for motion capture sequences used to build animations\\\\
Displacement constraints for interactive modeling and animation of articulated structures -- 1994\\
fitting geometric constraints using physics\\\\
A constrained inverse kinematics technique for real-time motion capture animation -- 1999\\
Dancing-to-Music Character Animation

%\subsection{synthesis}
Synthesizing Dance Performance Using Musical and Motion Features -- 2006\\
between music and an animation generated from a motion graph built from motion capture

\chapter{Theoretical Foundations of Animation}

\section{Skeletons, Rigs, and Controllers}


\section{In-betweening}


\section{The Line of Action}


\section{Space-time Sketching}


\section{Optimization Problems}


\section{Rig Combinations as Trees}

\chapter{Extending the Line of Action to Multiple Characters}\label{chap:implementation}

\section{Approach}
Almost inverting the problem at hand, we start from a pose to find the appropriate lines of action by tracing over the shapes of the dancers' bodies using the Grease Pencil Tool in Blender. It was carried out with the purpose of clarifying the properties and limitations of the typical line of action. After seeing at which point the actions started to repeat, a shorter sub-clip was chosen. Annotation was performed in five different ways:
\begin{enumerate}
	\item tracking and marking each characters' contact with the ground
	\item tracking and marking each characters' contact with each other
	\item tracing lines of action for each character separately every few frames, using the previous line of action notation -- the baseline method.
	\item tracing lines of action for the characters and treating them as one using as few strokes as possible every few frames.
	\item once the main common combinations were established from the above two methods, tracing lines of action only for the selected extreme combination poses.
\end{enumerate}

\begin{figure}[h!]
	\centering
        \begin{subfigure}[b!]{0.45\textwidth}
        	\centering
                \includegraphics[width=\linewidth]{img/keyframe_case_7_baseline}
                \caption{The typical lines of action trace out the whole skeleton, and in the worst case is 3 lines per character.}
                \label{fig:baseline}
        \end{subfigure}
        \quad
        \begin{subfigure}[b!]{0.45\textwidth}
        	\centering
                \includegraphics[width=\linewidth]{img/keyframe_case_7_new}
                \caption{Treating the characters as one bigger character allows the user to draw fewer lines to indicate the same pose.}
                \label{fig:new_notation}
        \end{subfigure}%
        \caption{Comparing the old line of action and the new.}
	\label{fig:poses}
\end{figure}

Through this process, certain patterns and common poses emerged, enabling the development of an improved notation for illustrating the poses of characters during their interactions with each other. Patterns included symmetry, parallelism, and repetition. Of course, using these patterns and assumptions to our advantage is what drove the development of a solution. It seemed that during many instances in the clip, it would be more efficient to draw lines over the two dancers as if they were one character. This morphed ``combined'' character should be easier to pose and animate.

\begin{figure}[h!]
	\centering
        \begin{subfigure}[b!]{0.45\textwidth}
        	\centering
                \includegraphics[width=\linewidth]{img/parallelism}
                \caption{Parallelism.}
                \label{fig:parallelism}
        \end{subfigure}
        \quad
        \begin{subfigure}[b!]{0.45\textwidth}
        	\centering
                \includegraphics[width=\linewidth]{img/symmetry}
                \caption{Symmetry.}
                \label{fig:symmetry}
        \end{subfigure}%
        \caption{Patterns in the poses.}
	\label{fig:patterns}
\end{figure}

\begin{figure}[h!]
	\centering
        \begin{subfigure}[b!]{0.31\textwidth}
        	\centering
                \includegraphics[width=\linewidth]{img/keyframe_case_4_(3)}
                \label{fig:pose1}
        \end{subfigure}
        \quad
        \begin{subfigure}[b!]{0.31\textwidth}
        	\centering
                \includegraphics[width=\linewidth]{img/keyframe_case_10_(5)}
                \label{fig:pose2}
        \end{subfigure}%
        \quad
        \begin{subfigure}[b!]{0.31\textwidth}
        	\centering
                \includegraphics[width=\linewidth]{img/keyframe_case_7_(4)}
                \label{fig:pose3}
        \end{subfigure}%
	\caption{A selection of tracings over the dance clip from ``The Band Wagon.''}
	\label{fig:tracing}
\end{figure}

\subsection{Notation}
Taking inspiration from Sutton Notation, we propose a new notation for representing the poses of two characters both together and separately.

There is a dance scene in the film ``The Band Wagon,'' where Cyd Charisse and Fred Astaire start out by walking together side by side, exchanging twirls until it morphs completely into a swing style dance. This scene is our use case for inventing a notation which extends seamlessly to more than one character.

To determine which poses for these dancers were common, I annotated the video, keeping track of contact points between the two characters, the contact points with the characters and the ground, the main LOAs, and the secondary LOAs. Once the video was annotated, I categorized the keyframes into how many LOAs it took to create that pose. 

NOTE: I am going to add images and an explanation.

\ignore{
\begin{table}[!htb]
  \centering
  \begin{tabular}{ | c | c || c | c | c | }
    \hline
     & Separate & \multicolumn{3}{c |}{Together}\\ \hline
    1 LOA 
    & &
    \begin{minipage}{.15\textwidth}
      \includegraphics[width=\linewidth, height=20mm]{img/01keyframe}
    \end{minipage} & & \\ 
    \hline
    2 LOA 
    &
    \begin{minipage}{.15\textwidth}
      \includegraphics[width=\linewidth, height=20mm]{img/2loa_separate_keyframe}
    \end{minipage}
    &
    \begin{minipage}{.15\textwidth}
      \includegraphics[width=\linewidth, height=20mm]{img/02keyframe}
    \end{minipage}
    &
    \begin{minipage}{.15\textwidth}
      \includegraphics[width=\linewidth, height=20mm]{img/03keyframe}
    \end{minipage} & 
    \\ \hline
    3 LOA 
    &
    \begin{minipage}{.15\textwidth}
      \includegraphics[width=\linewidth, height=20mm]{img/3loa_separate_keyframe}
    \end{minipage}
    &
    \begin{minipage}{.15\textwidth}
      \includegraphics[width=\linewidth, height=20mm]{img/04keyframe}
    \end{minipage}
    &
    \begin{minipage}{.15\textwidth}
      \includegraphics[width=\linewidth, height=20mm]{img/05keyframe}
    \end{minipage} 
    & 
    \begin{minipage}{.15\textwidth}
      \includegraphics[width=\linewidth, height=20mm]{img/06keyframe}
    \end{minipage} 
    \\ \hline
    4 LOA 
    &
    \begin{minipage}{.15\textwidth}
      \includegraphics[width=\linewidth, height=20mm]{img/4loa_separate_keyframe}
    \end{minipage}
    &
    \begin{minipage}{.15\textwidth}
      \includegraphics[width=\linewidth, height=20mm]{img/07keyframe}
    \end{minipage}
    &
    \begin{minipage}{.15\textwidth}
      \includegraphics[width=\linewidth, height=20mm]{img/08keyframe}
    \end{minipage} 
    & 
    \begin{minipage}{.15\textwidth}
      \includegraphics[width=\linewidth, height=20mm]{img/09keyframe}
    \end{minipage} 
    \\ \hline 
    &
    \begin{minipage}{.15\textwidth}
      \includegraphics[width=\linewidth, height=20mm]{img/4-1loa_separate_keyframe}
    \end{minipage}
    & & & 
    \\ \hline
    5 LOA 
    &
    \begin{minipage}{.15\textwidth}
      \includegraphics[width=\linewidth, height=20mm]{img/5loa_separate_keyframe}
    \end{minipage}
    &
    \begin{minipage}{.15\textwidth}
      \includegraphics[width=\linewidth, height=20mm]{img/10keyframe}
    \end{minipage} & & 
    \\ \hline
    6 LOA 
    &
    \begin{minipage}{.15\textwidth}
      \includegraphics[width=\linewidth, height=20mm]{img/6loa_separate_keyframe}
    \end{minipage}
    & & & 
    \\ \hline
  \end{tabular}
  \caption{Types of Keyframes for Two Dancing Characters}
  \label{table:LOAChart}
\end{table}
}


\section{Merging Kinematics Trees}
Rather than changing the whole LOA concept and matching algorithm, we reduce the more complex problem of using LOAs on multiple characters to the original LOA problem and utilize the same iterative method to pose.

From \citep{guay2015space}, since there is a viewing plane constraint on which the user draws the posing line, only one angle per joint has to be calculated to get from its current position to the target position. Starting from one end of the selected body line, local joint rotations are found using these equations:

\begin{equation}\label{eq:matching}
\theta_i(t) = \angle (Px_{i+1}(t) - Px_i(t), z_{i+1}(t)z_i(t)) 
\end{equation}

where $Px$ is the joint directed projected onto the viewing plane at time $t$ and $z$ is the drawn joint direction determined by the LOA. The current joint $i$ is the parent of joint $i+1$. Then the angles $\theta_i$ are applied to each joint to move it to the new pose. This method is fast and performs well. So instead of changing it, we merge kinematic trees of characters into one combined tree. 

Internally, both the tree and graph structures of characters' skeletons are stored. There is also another type of node called a ``separator'' node, which is purely virtual and acts as a sort of local root to allow for more body line options.

\begin{figure}[!h]
\centering
\includegraphics[scale=0.3]{img/shoulder}
\caption{In between the spine and the shoulder joints is a separator node, so the user can select only the arm as a body line.}
\end{figure}

\begin{algorithm}[H]
 \KwData{list of character hierarchies $h$, list of corresponding joints $pairs$}
 \KwResult{a single merged hierarchy}
 \If{$pairs$ is empty}
 {
  create a root node $mega\_root$\;
  \ForEach{hierarchy in $h$} 
  {
   make root of hierarchy a child of $mega\_root$\;
   \Return $mega\_root$\;
  }
 }
 \ForEach{correspondence in $pairs$}
 {
  \ForEach{joint in correspondence (2)}
  {
   go up hierarchy to find closest root or separator\;
   \If{roots are same}
   {
    connect joints into a $mega\_node$\;
    union the two nodes children\;
   }
   \Else{
	create a root node $mega\_root$\;
	make each joint a child of $mega\_root$\;
	reverse edges from each joint to their respective original nodes\;
   }
  }
 }
 enumerate cycles in using Johnson's algorithm (\citep{johnson1975finding})\;
 make each cycle into a new $mega\_node$\;
 \Return $mega\_root$\;
 
 \caption{The $merge\_hierarchies$ function.}
\end{algorithm}

A combined node is made by averaging frames (translation and rotation) of other nodes, effectively becoming a rigid body. A combined node made from a cycle averages all the frames of the nodes in the cycle with higher weight on the combined nodes within the cycle. A combined node made by connecting just two nodes averages the two frames and adds a virtual root. By combining nodes this way, we convert the two characters' trees into one, which is the goal.

\section{Interface}
To select a body line for one character, the user clicks on which model they want either in the viewport or in the node list on the right hand side panel and activates annotation mode. Then they can press and hold shift and draw over the character, tracing the limb they want to select. A path is found between the two closest controllers to the endpoints of the LOA. If a path is found, the blue line shows what was finally selected. The red line is what the user drew and the green line shows a posing line.

\begin{figure}[h!]
	\centering
        \begin{subfigure}[b!]{0.45\textwidth}
        	\centering
                \includegraphics[width=\linewidth]{img/ui}
        \end{subfigure}
        \quad
        \begin{subfigure}[b!]{0.45\textwidth}
        	\centering
                \includegraphics[width=\linewidth]{img/ui2}
        \end{subfigure}%
        \caption{Selecting a body line.}
	\label{fig:selection}
\end{figure}

To pose a character after a proper body line is selected, the user holds the control key while drawing a line with the mouse and \autoref{eq:matching} is used to match the body line to the LOA after the user releases the mouse. The green line shows the LOA. Note that now in the node list, all joints contained in the body line are selected, which allows the user to set keyframes using the LOA.

\begin{figure}[h!]
	\centering
        \begin{subfigure}[b!]{0.45\textwidth}
        	\centering
                \includegraphics[width=\linewidth]{img/ui3}
        \end{subfigure}
        \quad
        \begin{subfigure}[b!]{0.45\textwidth}
        	\centering
                \includegraphics[width=\linewidth]{img/ui4}
        \end{subfigure}%
        \caption{Posing a body line.}
	\label{fig:posingbl}
\end{figure}

While in annotation mode, the user can then enter a sub-mode called joint selection mode. They can then click on pairs of joints and if valid, they will be highlighted in yellow. The user can deselect a joint before it goes into a pair, but when two joints are selected they are automatically put into a pair.

\begin{figure}[h!]
	\centering
        \begin{subfigure}[b!]{0.45\textwidth}
        	\centering
                \includegraphics[width=\linewidth]{img/ui5}
        \end{subfigure}
        \quad
        \begin{subfigure}[b!]{0.45\textwidth}
        	\centering
                \includegraphics[width=\linewidth]{img/ui6}
        \end{subfigure}%
        \caption{Selecting a pair of joints to connect.}
	\label{fig:correspondence}
\end{figure}

\chapter{Experimental Validation of Solution}\label{chap:results}

\section{Carefully Selecting Sample Keyframes}


\section{Establishing a Baseline for Comparison}


\section{Our Solution}

\subsection{Experiment}
Describe the performance metrics, experimental hypotheses, experimental conditions, test data,
and expected results. Provide the test data. Interpret the results of the experiments. Pay special
attention to cases where the experiments give no information or did not come out as expected.
Draw lessons and conclusions from the experiments. Explain how additional experiments could
validate or confirm results.

\subsection{Results}


\subsection{Conclusions and Future Experiments}

\chapter{Limitations and Discussion}\label{chap:discussion}
Discussion lessons learned from the experiments, and new problems that are raised. 
\chapter{Conclusion}\label{chap:conclusion}
The goal of the thesis was to extend recent research at IMAGINE, restricted to the animation of a single character, to the challenging case of several interacting characters. While the capability to animate a single character by drawing a curve is present, there is no established sketch-based way to animate multiple characters either interacting or moving separately. The sketch-based animation of dancing couples captures the challenge of animating two characters both separately and together.

First, I observed and analyzed the current methods of creating character animations with and without interactions, from sketch-based animation to generating animation through motion graphs made from existing data. I looked at the way artists, cartoonists, and choreographers draw lines to indicate motion, specifically that of dancers.

I have identified what I think those tools lack for the task of posing dancing characters. Based on my findings, I created an algorithm for being able to describe and reproduce the coordinated movements of two dancers during a dance step. I evaluated the current line of action tool with metrics that could later be compared to the combined kinematic tree approach.

Dance is simply an introductory example for the design and evaluation of the proposed techniques, but the methods have the possibility to be applicable more generally.
\appendix \chapter{Appendix}\label{chap:appendix}

\section{Glossary}

\begin{itemize}[\null]

\item \textbf{articulated character} \\
definition

\item \textbf{CCD} \\
cyclic coordinate descent

\item \textbf{constraint} \\
cyclic coordinate descent

\item \textbf{controller} \\
cyclic coordinate descent

\item \textbf{keyframe} \\
cyclic coordinate descent

\end{itemize}
%=========================================================


%=========================================================
\backmatter
\bibliographystyle{plain} 
\bibliography{bibfile}

\bibliographystyleonline{plain}
\bibliographyonline{online}

%\cleardoublepage % Goes to an odd page
%\pagestyle{empty} % no page number
%~\newpage % goes to a new even page

\end{document}